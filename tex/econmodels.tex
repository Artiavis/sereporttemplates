\section{Economic and Mathematical Models}

\subsection{Perfect Competition}

One important concept of the stock market is the idea of perfect competition. Perfect competition is a theory in economics that states that it is not possible for any one participant to have enough resources to control the entire market. In terms of our project, that boils down to the following:
\begin{enumerate}
\item
One single person cannot control the stock market.\cite{inv:pcomp} 
\item
Anyone should be able to enter or exit the market with ease.
\item
Buyers know the full details of any stock they are to trade.
\item
There is no difference in the buying and selling price.\cite{wiki:pcomp}
\end{enumerate}
The problem with this model is that it is not entirely perfect or plausible. In order for no one person to be able to control the stock market, we need to be able to control the amount of money someone can have, so that they do not have enough to make an effect like that. Although this is more difficult in the real world, we will only have to set a cap on the money in our system, a point where no one will be able to go above that is still not low enough that it would happen often. This will solve the problem so that one person will not be so ridiculously rich that they will overpower our site. Point two is taken care of by the solution of point one. No one person can affect the market dramatically, meaning that no one person would make the game any different by moving in or out of the market. Point three will be satisfied as long as we provide necessary details in our user interface so that one will know everything they have to know before they buy. This aspect will be satisfied by Yahoo Finances being they provide all the necesary details because they must live under these guidelines as well. The last point is one that is flawed because it is an actual issue that we will have a different solution for. It is described more in the next section where we talk about the bid-ask spread.

\subsection{Bid-Ask Spread}

The bid-ask spread is the term in the stock market that describes the difference in price of the bid and the ask. These two prices are marginally different, but always with the ask being the more expensive of the two.\cite{inv:bidask} There are many factors that cause the bid-ask spread, the most common one being the human compensation factor, something that we will not have to deal with in our environment. If you buy from a broker, a part of the payment is going to go back to that stock broker because they need a way to make money themselves.\cite{inv:spreaddet} Another way the human factor changes the game of the stock market is that when the prices are changing dramatically for any certain company, they will increase the ask amount so that the market makers get a higher profit. The bid-ask spread can also be increased if the stock price is very low. The reason for this is that, if its price is low, then that probably means that not many people are buying it and it's not going to make them much profit. These couple facts together can cause a fairly large bid-ask spread. Although this happens naturally in the stock market, we decided to not factor it in. We decided to always assume the bid-ask spread is zero dollars, that being that the bid and the ask are of equal amount, for the sake of keeping this project more simple. 

\subsection{Reporting Abuse}

There are many algorithmic approaches to the functionality of reporting abuse. We decided to make reporting in a very simple manner, for the sake of keeping this part of the project more lightweight. When someone is reported by a user, it is put on a notification list of the admins. These notifications are listed in a database, listing the users along with the reason for the report. If enough notifications are given, the status of the user's account is at the descretion of the admin. Some possible tracks an admin can take when they run into this problem would be to message the user in order to try to come to an agreement, ban the user forever or take a deduction from the account as a warning to show that they have done wrong. Our reason for this model is that rather than relying on an algorithm is that it will take the pain off of us, the programmers. There are many factors in the process of creating an algorithm to deal with reported users, and leaving it up to an admin to personally solve the problem is a much simpler solution that having to deal with all the different factors that could be taking part in the process. This solution won't always solve the problem. For example, if the site becomes a large success, there will either have to be many admins working on the site or the algorithmic approach will have to be implemented, but it will do for our purposes to keep it simple.

