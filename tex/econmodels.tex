\section{Economic and Mathematical Models}
\label{econmodels}
\subsection{Perfect Competition}

One important concept of the stock market is the idea of perfect competition. Perfect competition is a theory in economics that states that it is not possible for any one participant to have enough resources to control the entire market. In terms of our project, that boils down to the following:
\begin{enumerate}
\item
One single person cannot control the stock market.\cite{inv:pcomp} 
\item
Anyone should be able to enter or exit the market with ease.
\item
Buyers know the full details of any stock they are to trade.
\item
There is no difference in the buying and selling price.\cite{wiki:pcomp}
\end{enumerate}
The problem with this model is that it is not entirely perfect or plausible. In reality, exceptionally wealthy individuals can dominate entire sectors of the market; entering or exiting markets is hindered by commission charged by brokers; certain traders may know more about certain stocks than other traders (also known as insider trading); and buying and selling prices differ, according to the bid-ask-spread. Nevertheless, our platform makes simplifying assumptions about the market to avoid most of these issues, and when unavoidable, compromises with the economic reality.

The economic assumption of perfect competition states that one person cannot control an entire market. This assumption is reasonable for normal investors with significantly less capital than the market capitalization of companies or markets, and therefore it is reasonable to assume that in our platform, individuals cannot shift the market price of stocks by their participation. However, in reality, exceptionally wealthy individuals may have more assets than the market capitalization of certain small and even medium size corporations. If such an individual were to attempt to enter or leave a market suddenly, the entire market would experience a shift. To avoid the complication of having to model the effect of such actions by such parties (which is exceptionally unusual and not the intended purpose of this simulation platform), we constrain users' initial seed capital to be below a certain level to prevent them from achieving this level of market domination.

The assumption of being able to freely enter or exit a market is somewhat unrealistic when it comes to the stock market. In general, brokers charge commission to execute any trade on behalf of an investor, which contradicts the stipulation of freedom to enter or exit. Nevertheless, we constrain users' seed capital to be above a certain level, to the point where commission should be nearly negligible.

Though it is impossible to resolve an issue of information disparity (the very nature of which stems from third party sources), we make the assumption it is a non-issue. We assume that all users gain all their information exclusively from the information exposed by our Research tools. 

\subsection{Bid-Ask Spread}

Similarly, it is difficult to challenge the bid-ask spread, the difference between the sale and purchase price listed for a stock because the many underlying factors. \cite{inv:spreaddet} In certain circumstances, a fairly large bid-ask spread can occur. Although this happens naturally in the stock market, we do not factor it in. We make the simplifying assumption that the bid-ask spread is zero dollars, that being that the bid and the ask are of equal amount, with the ability to program that functionality into future versions of the platform. 

\subsection{Reporting Abuse}

There are many algorithmic approaches to the functionality of reporting abuse. We decided to make reporting in a very simple manner, for the sake of keeping this part of the project more lightweight. When someone is reported by a user, it is put on a notification list of the admins. These notifications are listed in a database, listing the users along with the reason for the report. If enough notifications are given, the status of the user's account is at the descretion of the admin. Some possible tracks an admin can take when they run into this problem would be to message the user in order to try to come to an agreement, ban the user forever or take a deduction from the account as a warning to show that they have done wrong. Our reason for this model is that rather than relying on an algorithm is that it will take the pain off of us, the programmers. There are many factors in the process of creating an algorithm to deal with reported users, and leaving it up to an admin to personally solve the problem is a much simpler solution that having to deal with all the different factors that could be taking part in the process. This solution won't always solve the problem. For example, if the site becomes a large success, there will either have to be many admins working on the site or the algorithmic approach will have to be implemented, but it will do for our purposes to keep it simple.

