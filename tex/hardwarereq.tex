\section{Hardware Requirements}

The hardware requirements for Capital Games are minimal
on the client side, and moderate on the server side.

\subsection{Internet Connection}

The server needs to have an internet connection. Because
all data are transmitted as text, it is technically possible
for the server to function on even a low-bandwidth connection.
Obviously this is not ideal and low bandwidth can increase
server latency during peak use hours. 

\subsection{Disk Space}

Under the current configuration, Capital Games does not commit
any additional resources to the server's disk storage during
runtime. Rather, all data are stored to memory, and only
backed up to the disk. Therefore, the disk requirements for
Capital Games is simply the sum of the storage occupied by
all program instructions for the system, or approximately
1GB at the time of this writing.

\subsection{System Memory}

Because all runtime data are stored to the server's memory, as
well as the space in memory occupied by the actual system
runtime, having a large amount of ``headroom'' is vital to the
performance of the application. Although it is hard to
analyze performance requirements of an application that is
still in active development, empirical evidence from users of
similar technology make a few key observations. First, the
amount of memory consumed by an idle application can work
out to be over 100MB. Next, the active application will load
copies of its database-stored information into memory in order
to operate over it, which can result in large spikes in memory
usage. Finally, operating over the loaded data itself can 
consume a large amount of memory. This is in addition to any
memory occupied by the databases and worker processes \cite{so:railsmem}.
Therefore, having at least 200MB should be the minimum required
for internal testing of our application. Obviously, increasing
user base will exponentially increase the memory requirements
of our application.

\subsection{Client-side Hardware Requirements}

The user needs to have an internet connection in order
to interact with the server remotely. Although the intended
use of the system entailing the use of a graphical web browser
strongly encourages the use of a monitor (as mentioned
previously, the responsive nature of the application means
that screen resolution is not a limiting factor), it is also
possible for technically proficient users to interact with the
server through its RESTful resources. At some future date,
we may publish the official RESTful API for Capital Games, but
at this point, interacting purely through a command line
interface is discouraged.