\section{On-Screen Appearance Requirements}
{
\begin{figure}
\centering
\includegraphics[width=5.5in]{./img/responsiveenough.jpg}
\caption{This mockup illustrates the concept of Responsive Design with our initial UI mockup. The Responsive Design provides a single unified page which reflows content to match the size of various user devices. Preliminary design elements include interactive graph, scrolling marquee, and persistent navigation bar, corresponding with ST-4, ST-7, and ST-9.
}
\label{ui:mockup}
\end{figure}
}
%\vspace

The on-screen appearance requirements fall into three general areas including utilizing responsive design, conforming to most popular screen resolutions and refraining from the use of non-universally supported client-side technologies. \cite{wiki:responsive} As more and more devices are becoming capable of browsing the web, one of the main on-screen requirements is to implement responsive client-side markup that can intelligently adapt to the client’s UI capabilities. As shown in Figure ~\ref{ui:mockup}, a single page intelligently redistributes its elements to provide a unified interface across devices (in this case, a desktop browser and a smartphone). These capabilities include screen size, screen resolution and input methods. With these points in mind, Capital Games will be built to be usable on traditional desktop browser environments as well as mobile platforms. Javascript will be used to determine the best presentation of a page depending on the user’s browser. 

While a number of standards are emerging in the mobile market in regards to standard screen resolutions, there is still great variability present in conventional monitor sizes and resolutions. According to w3schools.com, as of February 2013, less than 10\% of Internet users have a screen resolution less than 1024x768. Therefore, an additional on-screen appearance requirement make Capital Games usable with screen resolutions greater than or equal to 1024 x 768. Finally, client side technologies must also be restricted to ones that are universally supported. Adobe Flash technology will not be used, as it isn’t universally supported. Flash can also become pretty sluggish on the client’s browser. This constraint will most likely lead to faster content loading and a more fluid user experience. Instead, HTML5, CSS and Javascript will be used to facilitate interactivity and determine the most suitable presentation of content.