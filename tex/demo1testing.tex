\chapter{Unit \& Integration Testing}

Testing is a critical part of software engineering -- just as important,
if not more so, than the actual programming. Without tests, how can one
know that anything works as intended?

There are two types of testing we intend to perform on our project: unit
and integration testing. Both live within the ``spec'' folder of our
directory.

\section{Unit Testing}

Because we programmed our application using Ruby, we programmed
our unit tests using the RSpec programming suite. RSpec provides
a natural and domain-specific language for testing individual suites
of functionality, both within generic Ruby, and especially within
Ruby on Rails web frameworks such as ours. Our tests check the 
responsiveness of both individual attributes of different modules,
and even that they behave in controlled manners. 

\section{Integration Testing}

Integration testing is always more difficult than unit testing, because
of the need to test as many interactions as possible between units. 
To simplify this, we will leverage the Capybara testing suite, which
utilizes the Selenium Web Driver to simulate end-user interactions
and test the expected output against the actual results of the
interactions. Due to time constraints, we were unable to author
integration tests for the application, which is something we intend
to remediate promptly.