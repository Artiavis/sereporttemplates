One more data structure that will be implemented for our system is a tree. As described earlier, the finance adaptor will need to make use of the information on EoDdata so that companies can be validated for existance before going through a trade. EoDdata does not come with a simple solution to find out if a single company is in existance and neither does Yahoo! Finance, therefore we must build a function that will do this for us. We could scan through every company on EoDdata everytime we need to validate, but that would waste too many resources. Instead, we decided to keep a local copy that will have very fast lookup of companies. The way that this will be implemented is to keep a tree in which the nth level of the tree represents the nth letter of the company symbol. For example, if the company with symbol ``GOOG'' exists, the head will point to G, which will point to O and so on. The last letter in the symbol will also have a boolean value to denote that this is the end of a symbol so that there could be companies with the same letters but one with an extra letter at the end. The reason for using a tree is because it will have a time complexity equal to the length of the symbol, which is a very small value, and a space complexity much smaller than if we used a structure such as a hash table. All we need for this tree is the ability to add a symbol, remove a symbol and check if a symbol exists. With these three simple commands, we can create our tree and maintain it to stay up-to-date.
