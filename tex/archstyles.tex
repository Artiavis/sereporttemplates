\section{Architectural Styles}

Capital Games was designed to conform with several well-established
software design principles. 
Some were chosen because of the software technologies
employed (ie an MVC-based web framework), others 
represent a natural evolution of the needs of the 
system. 

\subsection{Model-View-Controller}

Our philosophy in designing our website is to maintain
a separation between the subsystems responsible for 
maintaining user information and those responsible
for presenting it, in comformation with modern software
engineering practice. 

Therefore, we employ the Model-View-Controller (MVC)
architecture pattern. In MVC, a View requests from the
model the information it needs to generate an output;
the Model contains user information; and the Controller
can send commands to both the views and the models. \cite{wiki:mvc} 

This approach has made site design easier,
by abstracting the interface specifications from the 
system responsibilities. The Views and Models each know 
only what they need, while the Controller and associated
subsystems perform all the ``business logic''. The only
complexity added by the decision to employ MVC is that
updates to system components often have a ripple effect
and require numerous modifications elsewhere in the system.

\subsection{Representational State Transfer}

As a well-designed web application, Capital Games 
conforms with the universal practice of employing
RESTful design principles. RESTful design dictates,
amongst other constraints, that a platform have a
client-server relationship with the user (see below),
that the interface is uniform, and that all information
necessary for a request can be understood from the
request sent to the server. \cite{wiki:restful} 

We strive to keep the interface as uniform as possible
so that it is clear to the user how he is interacting
with Capital Games, on a multitude of levels. For example,
when purchasing a group of stocks, a user may graphically
``click on'' a submit button for a certain order, but
in effect he is also submitting an HTTP POST request with
appropriate form data to the Orders resource. 

This identificaiton of resources creates a tradeoff. On
the one hand, all RESTful architecture must be designed
at once, so that all resources are identified simultaneously,
and the state transfers are possible to each of them. On
the other hand, once resources are properly identified, the
distribution of responsibilities is trivial for every possible
interaction.

\subsection{Data-centric}

As a financial trading platform, Capital Games revolves
around user data. To simplify access to that information
from a variety of systems and to organize the data coherently
and with the possibility of rapid retrieval, we eventually store all user data in a relational database. In this way,
advanced queries can be performed on sets of data, both in 
application layer logic as well as by database administrators.
Additionally, storing user data outside of a particular
program's memory space enables subsystems which exist outside
of the current application layer to also have access to the
data. This additionally presents greater flexibility in terms
of scaling site infrastructure.


\subsection{Client-Server}

By its definition as a web application, Capital Games follows a client-server
model. The client, a user, interacts with the server, 
the various systems encapsulated by Capital Games.