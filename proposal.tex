\documentclass[11pt,letterpaper,oneside]{memoir}
\usepackage{mystyle}

% define custom title layout (slightly temperamental)
\title{%
{\color{color2} \hrule}\vspace{1cm}
\Huge{\color{color1} Project Proposal\\ % Specifications\\
{\emph{for}}\\
``Capital Games'' \vspace{1cm}
}
{\color{color2} \hrule}\vspace{1cm}
\Large{ \color{color2} Proposal\\
Software Engineering\\
14:332:452}
}

% define custom author layout (highly temperamental)
\author{\huge{\color{color1}Team 2:\\}\vskip.1in \Large{Jeff Adler \\Eric Cuiffo\\Nick Palumbo\\Jeff Rabinowitz\\Val Red\\Dario Rethage}}
\date{\Large{\today}}



\usepackage{hyperref}
\begin{document}
%\maketitle % don't use this, will break the Author section
\titleGM    % use this instead, defined to avoid problem


\pagebreak  % flush the next page
\tableofcontents % create TOC
%\chapter*{Revision History}
%\chapter{Introduction}
%\section{Purpose}
%\section{Project Scope and Product Features}
%\section{References}
%\chapter{Overall Description}
%\section{Product Perspective}
%\section{User Classes and Characteristics}
\chapter{Team Profile}
Team 2 would like to propose ``Capital Games'', a Stock Market Fantasy League.
This will be a variant of Project \#5 described on
\href{http://www.ece.rutgers.edu/~marsic/books/SE/projects/}{the course website}.
Further details are described in Chapter \ref{proposal}. This project is intended
to be a proof-of-concept, infusing new and exciting innovations in web
design and programming into a ``classic'' web application.

At this time we do not intend to nominate a Team Leader, as we believe we can 
divide our responsibilities evenly.

We follow with the profiles of our project team members.

\section{Jeff Adler}

%\subsection{Strengths}
Jeff is proficient with Java, C, C\#, C++, Objective-C, VB, Python, Ruby,
PHP, Delphi, Pascal, Actionscript, Silverlight, Coldfusion,
Eiffel, Bash, Ubercode, BeanShell, and Powershell. He is also experienced
with project presentation and user experience design.

\section{Eric Cuiffo}

%\subsection{Strengths}
Eric is proficient with C++, Python, HTML/CSS, and jQuery. He is 
familiar with Django and Android development as well as web design.

\section{Nick Palumbo}

%\subsection{Strengths}
Nick is proficient with C and Java. He is also familiar with Ruby on Rails,
HTML/CSS, Bash, and SQL, and his skills include user experience design.

\section{Jeff Rabinowitz}

%\subsection{Strengths}
Jeff is proficient with C++, and is also familiar with Ruby, Python,
C\#, and HTML/CSS. He also has experience with organization and management,
presentation, and content layout.

\section{Val Red}

%\subsection{Strengths}
Val is proficient with HTML/CSS, PHP, and C++. He is also familiar with
Ruby, Python, and Bash. He also possesses strong presentation and design
skills.

\section{Dario Rethage}

%\subsection{Strengths}
Dario is proficient with Java, PHP, C, Objective-C, and HTML/CSS. He
possesses strong organization and management as well as strong design skills.


\chapter{Project Proposal}
\label{proposal}
As mentioned previously, Team X would like to work on ``Capital Games''
(Project \#5 on 
\href{http://ece.rutgers.edu/~marsic/books/SE/projects/}{the course website}).
Our goal is to create a functional stand-alone Stock Market Fantasy League application
to serve as a platform for executing stock market simulations. We intend to deploy
to Amazon's Elastic Compute Cloud (EC2), a virtual cloud hosting provider. EC2 provides
a virtual scalable machine instance on Amazon's servers, which provides a platform to 
expand infrastructure if necessary.

\section{Market Models}

The fundamental data modeled by stock market fantasy leagues is the stock market. There
are many and varied options available to traders, the most fundamental of which are
buy, sell, and short. We fully intend to support these trading operations. Supporting 
more advanced trading features, such as trading on margin, is also a goal. 

Group 6 in 2012 implemented an interesting feature: user-defined mutual and hedge funds.
They created a model in which a single user could create a portfolio in which other 
users could invest, and follow the gains and losses of that fund. This exposes an interesting
notion, that of following the trades of other users in a given league. 

We therefore propose functionality of visit-able trader profile pages for 
users within a given league. In this model, should be able to track the trades 
and portfolio performances of their peers, while not being able to execute trades 
on their behalf. This should promote the competition of the league.

\section{Social Media Integration}

In recent years, some groups, such as Group 3 in 2012 and Group 6 in 2011, have provided
Facebook interfaces for their applications. These come in two variants: making Facebook
the \emph{de facto} portal for an application; and providing the option of using Facebook
as an authentication system. We note that Group 6 failed to achieve the Facebook authentication
by their deadline. Therefore, at this time we plan to strike a more conservative path and
create a stand-alone website with as few external dependencies as possible.

On the other hand, incorporating modular external features is a definite possibility. We note
that Group 3 created a Twitter interface for their trading application, through which users
were able to ``tweet'' trades. Such advanced functionality is an option, but should be considered
an additional feature, and as such we can consider adding this module at a later point.

\section{Transaction Feed}

Another feature inspired by Twitter and Facebook is the implementation of a user transaction
news feed. A feature which could be integrated into the ``social'' aspect of the leagues
is the ability to have a feed added to certain relevant pages, containing updates about 
which transactions other players are making. This could offer an interesting study as to how
players react to real-time information about player positions.

\section{Data Source}

After conducting market research, we decided to use Yahoo! Finance's HTTP 
interface for 20-minutes-delayed financial markets data, as have groups in 
previous years. Though other services
exist (e.g. 
\href{http://www.bloomberg.com/enterprise/enterprise_products/data_optimization/data_feeds/}{Bloomberg},
\href{http://www.financialcontent.com}{Financial Content}), they are nearly all paid-subscription models. 
The few services which offer free or ``freemium'' services (e.g. 
\href{http://eoddata.com}{eoddata}) have other unacceptable limitations, such as a 
lack of live data. Despite its age, Yahoo! Finance API is currently still the only 
free, versatile finance API. This is despite its unofficial ceiling of daily requests,
as we do not expect to receive sufficient volume of traffic as to approach this ceiling.

\section{Mobile and Tablet Interfaces}

One area in which we intend to differ from teams of previous years is with a unified
mobile and desktop website experience. Due to recent advances in web site design, 
known as ``responsive design'', it is possible to design a single site which is accessible 
from mobile, tablet, and traditional web browsers. Various CSS libraries provide
these designs styles, including
\href{http://twitter.github.com/bootstrap/}{Twitter Bootstrap}
and
\href{http://getskeleton.com}{Skeleton}.
Additionally, enhancements in mobile 
browser capabilities enable the use of Javascript, which is now supported universally
by all modern mobile browsers. These changes opens up availability of our team to dedicate 
more resources towards core site functionality.

\section{Interactive Portfolio Graphs}

Another area in which we intend to offer improvements over features provided in previous years
is that of interactive portfolio graphs. Highcharts JS provides the 
\href{http://www.highcharts.com/products/highstock}{HighStock} dynamic interactive graphing API.
This particular library is designed with financial modeling in mind, as it includes 
dynamic tooltips, time scrolling and time zooming. Thus we can present a user's entire 
portfolio's performance in a single object, for concise control and trend analytics. Other 
options include graphing propsective investments and graphing multiple investments on a 
single graph for comparison.

\section{Periodic Portfolio Email Updates}

A feature often presented to commercial investors is periodic performance update emails.
Like Group 3 in 2011, we would like to implement an email system which
will periodically inform users of their portfolio performance in various leagues, as well
as any gains or losses they may have incurred since the last update. We note that Group 3
failed to model performance of their portfolios, instead simply regurgitating the day's-end
values. An important improvement will be offering the ability to compare to previous trading
data.

\chapter{Subsystem Teams}

We have decided to divvy up the group into three principle subsystem teams, with additional
modules of functionality also assigned based on relevance. Because of the ``MVC'' logical
abstraction of the web framework we intend to leverage, Ruby on Rails, we intend to 
section our group accordingly, with each team focusing primarily on one component. Of course
teams will communicate and coordinate, as is necessary for such a large scale project.
However, this gives each developer a field of expertise and responsibility.

Each team is responsible for unit testing their own projects.

\section{Models Team}

Jeff Adler and Jeff Rabinowitz will be working on the Models Team. In MVC architectures, models
are abstractions which logically map database entries to programmatic objects. The Models Team
will be responsible for developing, testing, and debugging the implementations of databases,
object mapping, and data models. 

The Models Team will also be responsible for abstracting the Yahoo! Finance API so as to be 
accessible to the Views Team, who will use the results to query for live data. (This may
or may not include incorporating returned data into the database, to be worked out with the 
Views Team.)
Furthermore, the Models Team will configure the EC2 service, web servers, databases, 
deployments, and maintenance. 

\section{Views Team}

Dario Rethage and Nick Palumbo will be working on the Views Team. In MVC architectures, views
are abstractions of static interfaces with dynamically loaded content. The Views Team will be 
responsible for authoring and formatting the HTML/CSS of our responsive web design, as well
as incorporating the models to dynamically load content into a responsive design. This includes
building user and league forms, site navigation bars, and logging and trading interfaces.

The Views Team will also be responsible for building class(es) to parse queried Yahoo! Finance
data and translate it into the views. This includes the HighStock Javascript graphs, which
require data to render, as well as any tables or charts which may contain data textually. 

\section{Controllers Team}

Val Red and Eric Cuiffo will be working on the Controllers Team. In MVC architectures,
controllers are abstractions of user interactions with the site, called ``actions''. The 
Controllers Team will be responsible for defining the RESTful site design, as well as user
authorization and authentication systems. (There are Rails plugins which offer these 
utilities.)
They will also be responsible for implementing the
email updates system, although designing email templates and tracking subscriptions will be
handled by the Views and Models Teams, respectively.

\end{document}
